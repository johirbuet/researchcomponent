\section{Controbutions of the paper}
The authors of the paper rewrite visualization queries $Q$ using data reduction operator 
$M_R$ such that the visualization of the original data from query $Q$ and the visualization 
from the query $Q_R = M_R(Q)$ are similar and error free. 
\begin{figure}[h]
	\includegraphics[width=0.5\textwidth]{qr}
	\caption{Time series visualization: a) based on original data; b) Using data reduction operator;}   
	\label{fig:1}
\end{figure}
As shown in Figure \ref{fig:1} $Q_R$ produced the same visualization as $Q$ with almost 10 times less tuples and 10 times reduced time. 
The main contributions of the paper are following:
\begin{itemize}
	\item Proposed a visualization driven query rewriting technique relying on relational operators and parameterized with width and height of the desired visualization
	\item Focusing on the detailed semantics of the line charts, they propose a visualization driven aggregation strategy that only select necessary points needed for visualization. For visualization, in every time interval which corresponds to a pixel column in the visualization they select four tuples. The starting tuple, ending tuple, max tuple and the min tuple. 
\end{itemize}
\subsection{Query Rewriting}
TODO Query rewriting
%\subsection{Time Series Visualization}
TODO
\subsection{M4 Aggregation}
M4 is a value preserving aggregation strategy for time series data.
\begin{figure}[h]
	\includegraphics[width=0.5\textwidth]{m4}
	\caption{M4 query and visualization}   
	\label{fig:2}
\end{figure}
It divides the entire time series dataset into $w$ equal groups and thus each pixel column in the visualization takes only one group. For each group M4 select the aggregates $min(v)$,
$max(v)$, $min(t)$,$max(t)$ and that is why it is called M4 aggregation and then it joins the aggregated data to the time series and add missing timestamps $t_{bottom}$,$t_{top}$ and missing values $v_{first}$, $v_{last}$. In Figure \ref{fig:2} an example M4 query and 
the corresponding visualization is shown.

\textbf{Complexity of M4}: The grouping and computation of aggregated values can be done in $O(n)$ time where n is the number of tuples in the original query $Q$. Then the sub-sequent joining of the $4.w$ aggregated tuples with $Q$ requires $O(n+ 4.w)$ using hash join.
\subsection{M4 Upper Bound}

%\subsection{Time Series Data Reduction}

